\section{A Toy Model}
%
%
\begin{frame}
  So far we have worked out a sound theoretical foundation to understand the
  errors incurred when building learning algorithms.  In this section we will
  analyze in detail how these concepts look with a toy model.
\end{frame}
%
%
\begin{frame}
  Our data generating process will be very simple so that we can fully analyse
  the situation:

  \begin{align*}
    X &\sim U(0, 2 \pi) \\
    Y &\sim \sin(X) + N(0, \epsilon)
  \end{align*}

  Where $U$ is the uniform distribution on an interval, and $N$ is the normal
  distribution with a given mean and variance.
\end{frame}
%
%
\begin{frame}
  \begin{figure}
    \includegraphics[scale=0.09]{true_signal}
  \end{figure}

  Clearly, the regression function $E(Y \mid X)$ is given by:
  $$ \F(X) = E[ \sin(X) + N(0, \epsilon) \mid X ] = \sin(X) $$
\end{frame}
%
%
\begin{frame}
  The irreducible error component, which does not depend on our choice of
  learning algorithm, is easy to compute straight from the definition:

  \begin{align*}
      \IESE(x) &= E_Y \left[ \left( y - \F(x) \right)^2 \mid x \right] \\
      &= E_Y \left[ \left( \sin(X) + N(0, \epsilon) - \sin(X) \right)^2 \right] \\
      &= E_Y \left[ N(0, \epsilon)^2 \right] \\
      &= \epsilon
  \end{align*}
   
\end{frame}
%
%
\begin{frame}
  We take as our learning algorithm \textbf{linear regression}:
  $$ \LinReg: \D \mapsto \LinReg({\D}_{X}, {\D}_{Y}) $$
  \begin{figure}
    \includegraphics[scale=0.09]{single_fitted_line}
  \end{figure}
\end{frame}
%
%
\begin{frame}
  Let's study the bias of our toy model.  Recall the definition:
  \begin{align*}
    \BIAS (x)^2 = \left( \F(x) - Ef(x) \right)^2
  \end{align*}
  Where:
  \begin{align*}
    Ef(x) = E_D \left[ f(x; \D) \mid x \right] \\ 
  \end{align*}
  is the expected output of our modeling algorithm.
\end{frame}
%
%
\begin{frame}
  For our toy situation we can calculate $Ef$ numerically (I used scipy):
  $$ Ef(x) \approx -0.304 x +  0.955 $$
  The details of this computation are included in an appendix.
\end{frame}
%
%
\begin{frame}
  \begin{figure}
    \includegraphics[scale=0.12]{best_linear_fit}
  \end{figure}
\end{frame}
%
%
\begin{frame}
  The bias at a point is the square of the vertical distance between the true
  signal and the best linear fit.
  \begin{figure}
    \includegraphics[scale=0.08]{best_linear_fit}
  \end{figure}
\end{frame}
%
%
\begin{frame}
  The total bias is the expectation of the pointwise bias.  Visually, we can
  think of the unsigned area between the best linear fit and the true signal:
  \begin{figure}
    \includegraphics[scale=0.08]{model_bias}
  \end{figure}
\end{frame}
%
%
\begin{frame}
  The total bias can be explicitly calculated in this case (I used a numerical
  integration routine):
  $$ \BIAS^2 = E_X \left[ \left( \F(x) - Ef(x) \right)^2 \right] \approx 1.23
  $$
\end{frame}
%
%
\begin{frame}
  Bias can be lowered by making our learnign algoritm more complex.  For
  example, fitting a \textit{cubic} regression lowers the bias of our model
  considerably:
  \begin{figure}
    \includegraphics[scale=0.08]{cubic_model_bias}
  \end{figure}
  $$ \BIAS^2 = E_X \left[ \left( \F(x) - Ef(x) \right)^2 \right] \approx 0.028
  $$
\end{frame}
